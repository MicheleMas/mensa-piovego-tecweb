\documentclass[10pt,a4paper,onecolumn]{article}

\usepackage[utf8]{inputenx}
\usepackage[T1]{fontenc}
\usepackage{lmodern}
\usepackage{listings}
\usepackage{textcomp}
\usepackage[english,italian]{babel}
\usepackage{amsmath}
\usepackage{booktabs}
\usepackage{graphicx}
\usepackage[font=small,labelfont=bf,labelsep=period,tableposition=top]{caption}
\usepackage{tabularx}
\usepackage{multirow}
\usepackage{longtable}
\usepackage{fancyhdr}
\usepackage{lastpage}    
\usepackage{color}
\usepackage{enumitem}

\fancyhead{}
\renewcommand{\headrulewidth}{1pt}

\fancyhead[RE,RO]{
\begin{picture}(-135,0)
	\put(-475,0){\sffamily\large\leftmark}
\end{picture}
}

\cfoot{}

\fancyfoot[RO,LE]{\sffamily Pag.~\thepage{} di \pageref{LastPage}} 
\fancyfoot[RE,LO]{RistorESU Nord Piovego}

\renewcommand{\footrulewidth}{.2pt}
\pagestyle{fancy}

\renewcommand{\sectionmark}[1]{\markboth{#1}{#1}} 

% **************************************************
% Cross-references e collegamenti ipertestuali
% **************************************************
\usepackage[]{hyperref}
%\usepackage[hidelinks]{hyperref}
\hypersetup{%
  colorlinks=false, linktocpage=false, pdfborder={0,0,0}, pdfstartpage=1, pdfstartview=FitV,%
  urlcolor=Cyan, linkcolor=Cyan, citecolor=Black, %pagecolor=Black,
  pdfcreator={pdflatex}, pdfproducer={pdflatex with hyperref package}%
}

\definecolor{dkgreen}{rgb}{0,0.6,0}
\definecolor{gray}{rgb}{0.5,0.5,0.5}
\definecolor{mauve}{rgb}{0.58,0,0.82}
 
\lstset{ %
  %language=XHTML,                % the language of the code
  basicstyle=\footnotesize,           % the size of the fonts that are used for the code
  numbers=left,                   % where to put the line-numbers
  numberstyle=\tiny\color{gray},  % the style that is used for the line-numbers
  stepnumber=2,                   % the step between two line-numbers. If it's 1, each line 
                                  % will be numbered
  numbersep=5pt,                  % how far the line-numbers are from the code
  backgroundcolor=\color{white},      % choose the background color. You must add \usepackage{color}
  showspaces=false,               % show spaces adding particular underscores
  showstringspaces=false,         % underline spaces within strings
  showtabs=false,                 % show tabs within strings adding particular underscores
  frame=single,                   % adds a frame around the code
  rulecolor=\color{black},        % if not set, the frame-color may be changed on line-breaks within not-black text (e.g. comments (green here))
  tabsize=2,                      % sets default tabsize to 2 spaces
  captionpos=b,                   % sets the caption-position to bottom
  breaklines=true,                % sets automatic line breaking
  breakatwhitespace=false,        % sets if automatic breaks should only happen at whitespace
  title=\lstname,                   % show the filename of files included with \lstinputlisting;
                                  % also try caption instead of title
  keywordstyle=\color{blue},          % keyword style
  commentstyle=\color{dkgreen},       % comment style
  stringstyle=\color{mauve},         % string literal style
  escapeinside={\%*}{*)},            % if you want to add LaTeX within your code
  morekeywords={*,...},              % if you want to add more keywords to the set
  deletekeywords={...}              % if you want to delete keywords from the given language
}

% **************************************************
% Macro
% **************************************************
\newcommand{\sitepage}[1]{\textcolor{cyan}{\textsf{#1}}}
\newcommand{\inglese}[1]{\foreignlanguage{english}{\itshape{}#1}}
\newcommand{\progname}[1]{\textcolor{blue}{\textsf{#1}}}

\begin{document}
%----------------------------------------------------------
\begin{titlepage}

\begin{center}
% Upper part of the page
 
\textsc{\Large}\\[5cm]

\includegraphics[width=0.4\textwidth]{Logo.png}\\[0.3cm]  
\noindent\rule{\textwidth}{0.4pt} \\[0.3cm]
\textsc{\Huge Progetto di}\\[0.25cm]
\textsc{\Huge Tecnologie Web}\\[0.3cm]
\textsc{\Large Sito ``RistorESU Nord Piovego''}
\noindent\rule{\textwidth}{0.4pt}\\[0.5cm]
\textit{``Sviluppare un sito secondo gli standard W3C e le direttive WCAG 2 AAA''} \\[0.5cm]
\textsc{23 febbraio 2014}\\[0.5cm]
\begin{minipage}{0.4\textwidth}
\begin{flushleft} \large
\emph{Studente:}\\
Claudio Guarisco\\
Daniele Ronzani\\
Gianluca Bariga Boscolo\\
Michele Massaro
\end{flushleft}
\end{minipage}
\begin{minipage}{0.4\textwidth}
\begin{flushright} \large
\emph{Matricola:} \\
1057761\\
1057310\\
1061301\\
1057513\\
\end{flushright}
\end{minipage}
\end{center}
\end{titlepage}
%-----------------------------------------------------------------------

\clearpage

\tableofcontents

\clearpage 

\begin{abstract}
Questo progetto consiste nella realizzazione di un sito per la mensa 'RistorESU Nord Piovego'.
Si tratta di un sito che permette di ottenere informazioni utili per tutti gli utilizzatori del servizio di ristorazione della mensa Piovego, come ad esempio la lista dei piatti serviti, le informazioni sulle tariffe o la posizione geografica.
\end{abstract}

\clearpage

\section{Analisi dei requisiti}
Il sito \'e stato progettato per avere come target gli studenti universitari, che rappresentano la maggior parte degli utilizzatori del servizio di ristorazione della mensa Piovego. Inoltre si \'e cercato di capire quali informazioni siano di maggior interesse, in modo da renderle semplici e veloci da reperire.
Dato che nel nostro bacino potenziali di utenti sono presenti anche molti studenti provenienti da altre nazioni, abbiamo pensato di inserire anche la lingua inglese in modo da renderlo comprensibile anche per loro. \\
Dato che \'e un sito la cui maggior parte degli utilizzatori sono giovani, si \'e cercato di utilizzare una grafica semplice e colorata. Il colore principale (giallo) \'e stato scelto per richiamare gli elementi cromatici presenti nella sede fisica, in modo da rendere coerente il sito con ci\'o che rappresenta.

\section{Design}

\subsection{Layout}

Il layout consiste quasi completamente in un layout a tre pannelli, in cui possiamo in ogni pagina i seguenti elementi:
\begin{itemize}
 \item \textbf{Header}: parte superiore che contiene il titolo della pagina e l'indicazione della lingua.... %TODO 
\end{itemize}

\subsection{Struttura organizzativa}

La struttura organizzativa utilizzata dal sito \'e gerarchica, ed \'e stata scelta per permettere all'utente di orientarsi facilmente. A tale scopo \'e stato mantenuta una profondit\'a media molto bassa, mantenendo anche una ridotta ampiezza utilizzando solo sei voci nel men\'u, in modo che l'utente possa facilmente creare una propria mappa mentale e possa evitare il disorientamento.
Inolte si \'e cercato di mantenere sempre presenti alcune informazioni importanti, che permettano di rispondere alle principali informazioni %parlare dei 6 assi W




\section{Analisi generale dell'accessibilit\'a}

Riguardo l'accessibilit\'a si \'e scelto di rispettare le direttive WCAG 2 AAA, e alcune delle scelte pi\'u importanti verranno descritte di seguito.

\subsubsection{Colori}

Come anticipato precedentemente, la scelta cromatica \'e stata scelta in base ai colori che rappresentano al meglio la mensa Piovego. Prima di applicare i vari colori al sito, \'e stato controllato che mantenessero sempre un contrasto molto alto, in modo da non creare alcun problema di accessibilit\'a. Inoltre in tutto il sito non sono mai state veicolate informazioni tramite colori, per evitare che gli utenti ipovedenti, o con problemi di daltonismo, non riescano ad accedervi.
%parlare della riga nera affianco al menu




\clearpage
\section{Mappa del sito}
\includegraphics[width=.9\textwidth]{mappasito.png}


\clearpage

\section{Accessibilità}\label{sec:accessibility}

\subsection{Il testo}\label{sec:testoaccessibile}


\subsection{I link}

\subsection{Le immagini}

\subsection{I colori}


\subsection{I form}


\subsection{La tabella}


\subsection{Screen reader}

\clearpage

\section{Strumenti ultilizzati}\label{sec:strumenti}

\section{Pagine dinamiche}

\subsection{XML}

\subsection{Perl}

\clearpage

\section{Norme di sviluppo}

\subsection{XML}


\subsection{CSS}


\end{document}


